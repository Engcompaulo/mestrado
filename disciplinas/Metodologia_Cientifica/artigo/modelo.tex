
%% modelo.tex
%% V1.0
%% 03/05/2009
%% Hugo Vieira Neto
%%
%% Modelo de artigo de conferência IEEE para LaTeX
%% Requer os arquivos IEEEtran.cls, IEEEtran.bst e IEEEabrv.bib

% *** PREÂMBULO ***
\documentclass[conference,peerreview]{IEEEtran}
% Na versão final do artigo, a opção peerreview deve ser removida para
% que as informações sobre os autores fiquem aparentes após o título.

% Pacotes para a utilização da língua portuguesa.
\usepackage[brazil]{babel}
\usepackage[utf8]{inputenc}
% As linhas acima devem ser comentadas se o artigo for escrito em inglês.

% Pacotes para o uso de equações e símbolos matemáticos.
\usepackage[cmex10]{amsmath}
\usepackage{amsfonts,amssymb}

% Pacotes para o uso de tabelas, gráficos e subfiguras.
\usepackage{array}
\usepackage{graphicx}
\usepackage[caption=false,font=footnotesize]{subfig}

% Pacote para a formatação ordenada automática de múltiplas citações.
\usepackage{cite}

% Pacote para a formatação adequada de URL.
\usepackage{url}

% Falhas de separação em sílabas devem ser listadas aqui
\hyphenation{}


% *** CORPO DO TEXTO ***
\begin{document}

% Título do artigo
\title{Aprendizado de máquinas e Espectroscopia Raman para identificação de fungos}
% Quebras de linha \\ podem ser inseridas para melhor formatação de títulos
% longos.


% Nome e afiliação do autor
\author{
	\IEEEauthorblockN{Daniel Silva Costa}
	\IEEEauthorblockA{
		CPGEI / UTFPR\\
		Avenida Sete de Setembro, 3165\\
		Curitiba-PR - CEP 80.230-910\\
		E-mail: eng.daniels.costa@gmail.com\\
		\url{https://github.com/danielscosta}
	} % \IEEEauthorblockA
} % \author
% Para múltiplos autores e afiliações, consultar exemplos no arquivo
% bare_conf.tex


% Formatação do título e autoria do artigo. Em artigos com a opção peerreview,
% a autoria é ocultada e as páginas são numeradas.
\ifCLASSOPTIONpeerreview
	\setcounter{page}{1}
	\IEEEpeerreviewmaketitle
\else
	\maketitle
\fi


% Resumo do artigo
\begin{abstract}
Aqui vai o resumo do artigo.
\end{abstract}

\section{Introdução}

\section{Revisão da literatura}

\section{Materiais e métodos}

\section{Resultados e discussão}

\section{Conclusão}

\section*{Reconhecimento}

\bibliographystyle{IEEEtran}
\bibliography{IEEEabrv,modref}

\end{document}