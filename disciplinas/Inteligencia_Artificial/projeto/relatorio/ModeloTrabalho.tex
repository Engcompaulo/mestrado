\documentclass[english]{article}
\usepackage[T1]{fontenc}
\usepackage[utf8]{inputenc}

\makeatletter \providecommand{\tabularnewline}{\\}

\usepackage{cec2003,multicol,times}
\usepackage{graphicx}
\usepackage{hyphenat}
\usepackage{indentfirst}
\usepackage{babel}
\makeatother

\begin{document}

\pagestyle{empty} 
\sloppy
\twocolumn[
\title{Análise do Espectro Raman de amostras de cachaça através de PCA e Redes Neurais Artificiais\\\vspace{0.5cm}}
\begin{center}
\textbf{Daniel S. Costa} \\
UTFPR\\
Av. Sete de Setembro, 3165 \\
80230-901 Curitiba, Brazil \\\vspace{0.5cm}
\end{center}
]

\begin{abstract}
O trabalho deverá ser em duas colunas de acordo com o modelo Latex aqui
apresentado (neste caso deverá ser limitado a 6 páginas) ou gerado
via outro editor (no caso de coluna simples limitado a 12 páginas).
Sugestões para as seções
estão  listadas a seguir. \vspace{2cm}
\end{abstract}

\section{Introdução}
\vspace{1cm}
 Breve introdução posicionando  o tema da pesquisa.
Identifique o contexto geral no qual o problema ou as questões da
pesquisa que foram identificadas. \vspace{1cm}



Para quem vai propor uma reprodução de resultados de referência da literatura, esta
seção deverá citar o artigo no qual o trabalho se baseia mas deverá ser diferente
do texto base.

\subsection{Descrição do Problema}
\vspace{1cm} O que será pesquisado? O que se vai fazer?
%\vspace{2cm}
\subsection{Motivação}
\vspace{1cm} Por que se deseja fazer a pesquisa? \vspace{1cm}

Identifique os argumentos que justifiquem que a sua pesquisa é
significativa, importante e/ou relevante. %\vspace{2cm}

\section{Revisão da Literatura}
\subsection{Descrição da técnica utilizada}
incluir nesta subseção uma descrição dos conceitos da(s) técnica(s)
que serão utilizados no seu trabalho.

\subsection{Descrição das abordagens relacionadas}
\vspace{1cm} Mostrar o enfoque recebido pelo tema já publicado na
literatura. \vspace{1cm}

Análise comentada do que já foi escrito sobre o tema procurando
mostrar os enfoques convergentes e divergentes dos diversos
autores. \vspace{1cm}

Para quem vai propor uma reprodução de resultados de referência da literatura, esta
seção deverá citar o artigo no qual o trabalho se baseia mas deverá ser diferente
do texto base.
Exemplo de citação das referências
bibliográficas~\cite{Russell:2003,Pedrycz:2007,haykin2001redes,Braga:2007}.
\vspace{2cm}
\section{Metodologia}
\vspace{1cm} Como sera realizada a pesquisa? \vspace{1cm}

Descrever detalhadamente a abordagem proposta. \vspace{2cm}
\section{Simulações e Resultados}
\vspace{1cm} É interessante que a abordagem possa ser simulada e
os resultados sejam apresentados nesta seção. Neste caso, o modelo
simulado baseado em uma técnica de IA deve ser avaliado e, se for o caso,
comparado com outro modelo, para verificar se há vantagem no uso
da técnica. \vspace{2cm}
\section{Conclusões}
\vspace{1cm} Esta seção deverá trazer as conclusões a respeito da
abordagem e resultados obtidos. \vspace{2cm}

\bibliographystyle{WCCI}
\bibliography{refIA}

\end{document}