\documentclass{article}
\usepackage[T1]{fontenc}
\usepackage[utf8]{inputenc}

\makeatletter \providecommand{\tabularnewline}{\\}

\usepackage{cec2003,multicol,times}
\usepackage{graphicx}
\usepackage{hyphenat}
\usepackage{indentfirst}
\usepackage[english]{babel}
\makeatother

\begin{document}

\pagestyle{empty} 
\sloppy
\twocolumn[
\title{Análise do Espectro Raman de amostras de cachaça através de PCA e Redes Neurais Artificiais\\\vspace{0.5cm}}
\begin{center}
\textbf{Daniel S. Costa} \\
UTFPR\\
Av. Sete de Setembro, 3165 \\
80230-901 Curitiba, Brazil \\\vspace{0.5cm}
\end{center}
]

\begin{abstract}
O presente trabalho valida a utilização de redes neurais artificiais na interpretação de dados do espectro Raman de amostras de cachaça. Combinando a modelagem a técnica PCA(Principal Component Analysis) de forma a maximizar a taxa de sucesso obtida. \vspace{2cm}
\end{abstract}

\section{Introdução}
\vspace{1cm} 
Este trabalho baseia-se no artigo [1] em que o espectro Raman de amostras de cachaças são submetidas a técnica PCA(Principal Component Analysis) de forma a separar as principais componentes da assinatura Raman tornando possível a visualização clusterizada de amostras contaminados por metanol ou não.

De forma complementar ao artigo citado, uma rede neural artificial foi treinada buscando obter a mesma classificação obtida pela técnica do PCA. Contudo, o número de amostras para o treinamento da rede é pequeno, sobretudo quando leva-se em consideração a quantidade de features que são 1024. Assim, observou-se que a rede neural não obteve a assertividade esperada, mesmo testando diferentes quantidade de neurônios na camada escondida da rede.

Outra abordagem adotada foi submeter as amostras ao PCA, e submeter os valores resultantes das 2 principais componentes a rede neural. Deste modo, o teste obteve uma taxa de sucesso de 80%.

Através dos experimentos realizados pode-se concluir que para uma grande quantidade de features e poucas amostras, a rede neural utilizada é ineficaz na classificação dos dados. Uma vez obtida as principais componentes da amostra através do PCA, a classificação se torna possível por conta da diminuição da quantidade de features avaliadas.

\vspace{2cm}
\subsection{Descrição do Problema}
\vspace{1cm} A Espectroscopia Raman é um técnica capaz de revelar importantes informações sobre a composição de um material analisado. Podendo, até mesmo, ser utilizado na classificação biológica, uma vez que, células de diferentes organismos apresentam diferentes composições químicas.
Contudo a interpretação da assinatura gerado pela medição do espectro Raman pode ser uma tarefa desafiadora, uma vez que determinar quais os picos da curva são relevantes na distinção de uma material de outro requer treinamento e experiência por parte de quem analisa tal assinatura.

\subsection{Motivação}
\vspace{1cm} Criar metodologias e automatizações na análise do dados da espectroscopia Raman pode vir a facilitar o processo de identificação e classificação de materias ou seres vivos. O que se mostra relevantes na busca de contaminantes em um produto ou até mesmo na identificação de uma bactéria/fungo num processo infeccioso.

\vspace{2cm}
\section{Revisão da Literatura}
\subsection{Descrição da técnica utilizada}
incluir nesta subseção uma descrição dos conceitos da(s) técnica(s)
que serão utilizados no seu trabalho.

\subsection{Descrição das abordagens relacionadas}
\vspace{1cm} Mostrar o enfoque recebido pelo tema já publicado na
literatura. \vspace{1cm}

Análise comentada do que já foi escrito sobre o tema procurando
mostrar os enfoques convergentes e divergentes dos diversos
autores. \vspace{1cm}

Para quem vai propor uma reprodução de resultados de referência da literatura, esta
seção deverá citar o artigo no qual o trabalho se baseia mas deverá ser diferente
do texto base.
Exemplo de citação das referências
bibliográficas.
\vspace{2cm}
\section{Metodologia}
\vspace{1cm} Como sera realizada a pesquisa? \vspace{1cm}

Descrever detalhadamente a abordagem proposta. \vspace{2cm}
\section{Simulações e Resultados}
\vspace{1cm} É interessante que a abordagem possa ser simulada e
os resultados sejam apresentados nesta seção. Neste caso, o modelo
simulado baseado em uma técnica de IA deve ser avaliado e, se for o caso,
comparado com outro modelo, para verificar se há vantagem no uso
da técnica. \vspace{2cm}
\section{Conclusões}
\vspace{1cm} Esta seção deverá trazer as conclusões a respeito da
abordagem e resultados obtidos. \vspace{2cm}

\section*{Referências}
\vspace{1cm} [1] R. E. De Góes, L. V. M. Fabris, M. Muller, and J. L. Fabris, “Light-
assisted detection of methanol in contaminated spirits,” Journal of
Lightwave Technology, vol. 34, no. 19, pp. 4499–4505, 2016.

\end{document}